\section{全同粒子}

\emph{Boson玻色子}:$s$为整数,波函数满足交换对称性。

\emph{Fermion费米子}:$s$为半奇数,波函数满足交换反对称性,因而有Pauli不相容原理。

一个常见的问题是计算Bose, Fermi与经典体系粒子的组合数:
$n$个全同粒子处于$m$个可能的单粒子态,对于Bose子, Fermi子与经典粒子,可能有几种组合方式?

对于经典粒子,粒子是完全可分辨的,因此粒子间互不相关,可能的组合数有
\begin{equation}
    N_{\rm{classical}} = m^n.
\end{equation}

对于Fermi子,粒子不可分辨,可以用隔板法求取可能的组合数:
设置$m-1$个隔板,与$n$个粒子混合排列,这样$n$个粒子就被隔板分至$m$个态上,有$(n+m-1)!$种方式。
隔板与粒子分别不可分辨,因此除以各自排列数$(m-1)!$与$n!$
\begin{equation}
    N_{\rm{Fermi}} = \frac{(n+m-1)!}{(m-1)! n!}.
\end{equation}

对于Bose子,粒子不可分辨,且是互斥的,于是从$m\ (m>n)$个态中选取$n$个分别放入一个粒子即可,可能的组合数:
\begin{equation}
    N_{\rm{Bose}} = P_m^n = \frac{m!}{n!(m-n)!}.
\end{equation}