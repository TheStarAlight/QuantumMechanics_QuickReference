\section{Landau能级}

粒子在电磁场中的Hamiltonian:
\begin{equation}
    H = \frac{1}{2m}\left(\pp-\frac{q}{c}\AA\right)^2 + q\phi,
\end{equation}
其中$\pp$是正则动量,$\AA$是磁场矢势,$\phi$是电势。
按电动力学,电场与磁场用势能表示为:
\begin{equation}
    \bm{E} = -\frac1c \pd_t \AA - \nabla \phi, \quad
    \bm{B} = \nabla \times \AA.
\end{equation}

考虑处于沿$z$轴的静磁场中的粒子,选取Landau规范
\begin{equation}
    A_x = -By, \quad A_y = A_z = 0,
\end{equation}
可得Hamiltonian:
\begin{equation}
    H = \frac{1}{2m}\left[ (p_x-qBy/c)^2 + p_y^2 + p_z^2 \right].
\end{equation}
鉴于$[p_x,H]=[p_z,H]=0$,$p_x,p_z$与$H$有共同本征态,取为
\begin{equation}
    \psi(x,y,z) = \ee^{\ii p_x x/\hbar} \ee^{\ii p_z z/\hbar} \phi_{p_x,p_z}(y),
\end{equation}
有
\begin{equation}
    H \phi(y) = \frac{1}{2m}\left[ (p_x-qB\hat{y}/c)^2 + \hat{p}_y^2 + p_z^2 \right] \phi(y),
\end{equation}
注意这里对算符$y,p_y$作了帽子记号,以区别于$p_x,p_z$(其作为$\phi(y)$的参数出现,是定值)。
置
\begin{equation}
    y_0 = \frac{p_x c}{qB}, \quad \omega = \frac{qB}{mc},
\end{equation}
于是Hamiltonian成为
\begin{equation}
    \frac12 m\omega^2 (y-y_0)^2 + \frac{1}{2m}p_y^2 + \frac{1}{2m}p_z^2.
\end{equation}
可见除去$z$方向动能,这是一个原点位于$y=y_0$的一维谐振子,能级表达式为
\begin{equation}
    E_n = \hbar \omega \left( n + \frac12 \right).
\end{equation}
