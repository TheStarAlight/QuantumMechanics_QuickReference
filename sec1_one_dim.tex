\section{一维问题}

% ===============
\subsection{无限深方势阱}

我们设无限深方势阱位于$0<x<a$,势阱中$V=0$,由驻波条件给出势阱中的波函数
\begin{equation}
    \psi_n(x) = \sqrt{\frac2a}\sin{k_n x}, \quad k_n = \frac{n\pi}{a},
\end{equation}
对应的能量为
\begin{equation}
    E_n = \frac{\hbar^2 k_n^2}{2m} = \frac{\hbar^2 \pi^2 n^2}{2ma^2}.
\end{equation}
注意:记忆能级不靠背诵,利用$p=\hbar k$及$E=p^2/2m$即可定出能级。

在求解如$x,p,x^2,p^2,H$等物理量的平均值时,注意:
\begin{itemize}
    \item 不少情况下,利用波函数的对称性及不同能级波函数的正交性可简化问题。
    \item 对于不变的势阱,$p,H$是守恒量,因此其平均值不随时间变化。
    \item 熟记二倍角公式:
    \begin{equation}
    \begin{aligned}
        \sin 2\alpha &= 2\sin\alpha\cos\alpha, \\
        \cos 2\alpha &= \cos^2\alpha - \sin^2\alpha = 2\cos^2\alpha - 1 = 1 - 2\sin^2\alpha.
    \end{aligned}
    \end{equation}
    \item 熟记积化和差公式:
    \begin{equation}
    \begin{aligned}
        2\cos\alpha\cos\beta &= \cos(\alpha-\beta) + \cos(\alpha+\beta), \\
        2\sin\alpha\sin\beta &= \cos(\alpha-\beta) - \cos(\alpha+\beta), \\
        2\sin\alpha\cos\beta &= \sin(\alpha+\beta) + \sin(\alpha-\beta).
    \end{aligned}
    \end{equation}
\end{itemize}

% ======================
\subsection{谐振子:坐标表象}

一维谐振子势为
\begin{equation}
    V(x) = \frac12 m \omega^2 x^2.
\end{equation}

置
\begin{equation}
    \alpha = \sqrt{\frac{m\omega}{\hbar}},
\end{equation}
解得谐振子的波函数为
\begin{equation}
    \psi(x) = \sqrt{\frac{\alpha}{\sqrt\pi 2^n n!}} H_n(\alpha x) e^{-\alpha^2 x^2/2},
\end{equation}
其中$H_n$为$n$阶Hermite函数。

能级为
\begin{equation}
    E_n = (n + \frac12) \hbar \omega.
\end{equation}


% ======================
\subsection{谐振子:Fock表象}

引入无量纲算符
\begin{equation}
    Q \coloneq \sqrt{\frac{m\omega}{\hbar}} x ,\quad P \coloneq \sqrt{\frac{1}{m\hbar\omega}} p,
\end{equation}
其对易子$[Q,P]=\ii$,
进一步定义算符
\begin{equation}
    a \coloneq \frac{1}{\sqrt{2}} (Q+\ii P) ,\quad a^\dag \coloneq \frac{1}{\sqrt{2}} (Q-\ii P),
\end{equation}
其对易子$[a,a^\dag]=1$,可得
\begin{equation}
    H = (a^\dag a + \frac12) \hbar\omega.
\end{equation}
由
\begin{equation}
    a^\dag a \ket{n} = n \ket{n}
\end{equation}
可见$a^\dag a$为谐振子的能级算符。

$a$和$a^\dag$分别对应谐振子的\emph{降算符}与\emph{升算符}。
由对易子$[a,a^\dag]=1$及Hamiltonian $H$的表达式可知:
\begin{equation}
    H a \ket{n} = (E_n - \hbar\omega) a \ket{n},\quad H a^\dag \ket{n} = (E_n + \hbar\omega) a^\dag \ket{n},
\end{equation}

$a$并非保留内积的幺正算符。
通过归一条件$\braket{n}{n}=\braket{n-1}{n-1}=1$可得
\begin{equation}
    a\ket{n} = \sqrt{n}\ket{n-1}.
\end{equation}
同理有
\begin{equation}
    a^\dag \ket{n} = \sqrt{n+1}\ket{n+1}.
\end{equation}
