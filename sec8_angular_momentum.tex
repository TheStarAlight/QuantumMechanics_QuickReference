\section{角动量}

% ===============
\subsection{升降算符}

升降算符定义为
\begin{equation}
    L_\pm = L_x \pm \ii L_y,
\end{equation}
从角动量算符对易关系可得升降算符满足的关系:
\begin{equation}
\begin{aligned}
    L_\pm L_\mp = L^2 - L_z^2 \pm \hbar L_z, \\
    L_x^2 + L_y^2 = L^2 - L_z^2 = \frac12 \left(L_\pm L_\mp + L_\mp L_\pm \right), \\
    [L_z, L_\pm] = \pm \hbar L_\pm, \\
    [L^2, L_\pm] = [L^2, L_z] = 0.
\end{aligned}
\end{equation}

升降算符对角动量态的作用效果:
\begin{equation}
\begin{aligned}
    L_+ \ket{lm} &= \sqrt{l(l + 1) - m(m + 1)} \ket{l, m+1},\\
    L_- \ket{lm} &= \sqrt{l(l + 1) - m(m - 1)} \ket{l, m-1},
\end{aligned}
\end{equation}
这里矩阵元的正负号可利用$L_+\ket{ll}=0$与$L_-\ket{l-l}=0$助记。

有时需要求解$L_x, L_y$的矩阵元,可利用
\begin{equation}
    L_x = \frac12 (L_+ + L_-), \quad L_y = \frac{1}{2\ii} (L_+ - L_-).
\end{equation}

% ===============
\subsection{角动量的耦合}

角动量的\emph{非耦合表象}:选取$L_1^2, L_2^2, L_{1z}, L_{2z}$力学量,以对应的$l_1,l_2,m_1,m_2$作为好量子数。
$m_1, m_2$的取值范围:
\begin{equation}
    -l_1 \le m_1 \le l_1,\quad -l_2 \le m_2 \le l_2.
\end{equation}

\emph{耦合表象}:选取$L_1^2, L_2^2, L^2=(\bm{L}_1 +\bm{L}_2)^2, L_z = L_{1z} + L_{2z}$力学量,以对应的$l_1,l_2,l,m$作为好量子数。
$l,m$的取值范围:
\begin{equation}
    \abs{l_1-l_2}\le l \le l_1+l_2,\quad -l \le m \le l.
\end{equation}

两个表象之间的变换矩阵元:
\begin{equation}
    \braket{l_1m_1l_2m_2}{l_1l_2lm} = C_{l_1m_1l_2m_2}^{lm},
\end{equation}
称为Clebsch-Gordan (CG)系数。
