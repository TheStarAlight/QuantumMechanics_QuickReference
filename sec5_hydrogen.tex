\section{中心力场中的氢原子}

中心力场中的氢原子,选取质心坐标和自然单位制,波函数为
\begin{equation}
    \psi_{nlm}(\rr) = Y_{lm}(\theta,\phi) R_{nl}(r),
\end{equation}

\begin{itemize}
    \item 角动量平方的期望值$\expval{\LL^2}=l(l+1)\hbar^2$;
    \item 角动量$z$分量的期望值$\expval{L_z}=m\hbar$。
\end{itemize}

计算$r^n$物理量的期望值时,可以采用如下的策略:
\begin{equation}
\begin{aligned}
    \expval{r^n}
    &= \frac{\int r^2 \dd r \dd\Omega \ Y_{lm}^*(\theta,\phi) r^n Y_{lm}(\theta,\phi) R_{nl}(r)^2}{\int r^2 \dd r \dd\Omega \ Y_{lm}^*(\theta,\phi) Y_{lm}(\theta,\phi) R_{nl}(r)^2} \\
    &= \frac{\int\dd r \ r^{n+2} R_{nl}(r)^2}{\int\dd r \ r^2 R_{nl}(r)^2},
\end{aligned}
\end{equation}
其中$R_{nl}(r)$是多项式$P(r)$与$\ee^{-r/na_0}$的乘积,因此问题最后可以归结为求解一个重要积分:
\begin{equation}
    \int_0^\infty \dd t \ t^{N} \ee^{-t} = \Gamma(N+1) = N!,
\end{equation}
读者应牢记。
