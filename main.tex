\documentclass[fontset=fandol, zihao=-4]{ctexart}
\linespread{1.5}
\usepackage[a4paper, scale=0.75]{geometry}
\usepackage{amsmath,amssymb,amsfonts}
\numberwithin{equation}{section}    % let the equation number displays in (1.1) style
\numberwithin{figure}{section}
\usepackage{unicode-math}
\setmathfont{STIX2Math}[Extension={.otf}, Path=./STIX2fonts/, Scale=1]
\setmainfont{STIX2Text}[Extension={.otf}, Path=./STIX2fonts/, UprightFont={*-Regular}, BoldFont={*-Bold}, ItalicFont={*-Italic}, BoldItalicFont={*-BoldItalic}]
\usepackage{graphicx}
\usepackage{xcolor}
\usepackage{hyperref}
\usepackage{physics}    % symbols related to physics
% \usepackage{ulem}       % strikethrough \sout
\usepackage{xeCJKfntef} % for Chinese underline style
\usepackage{tcolorbox}  % colored textbox environment
\tcbuselibrary{breakable}   % tcolorbox setup
\usepackage[angle=0, scale=0.5]{draftwatermark}
\SetWatermarkText{\includegraphics{figures/TheStarAlight_logo.png}}
\hypersetup{colorlinks=true,linkcolor=blue,citecolor=blue,urlcolor=magenta} % hyperref setup

% shortcuts
    % color
    \newcommand{\cyan}[1]{\textcolor{cyan}{#1}}
    % math style
    \renewcommand{\rm}[1]{\mathrm{#1}}  % roman, redefining original \rm
    \newcommand{\bm}[1]{\symbfit{#1}}   % bold+italic, using \symbfit instead of conventional \bm
    \newcommand{\br}[1]{\symbf{#1}}     % bold+roman
    % math
    \newcommand{\pd}{\partial}      % partial symbol
    % \newcommand{\dd}{\mathrm{d}}  % roman d (derivative & integral) [already implemented in the physics pkg]
    \newcommand{\ii}{\mathrm{i}}    % roman i (imag.)
    \newcommand{\ee}{\mathrm{e}}    % roman e (exp)
    % \newcommand{\abs}[1]{\lvert #1 \rvert}  % abs symbol [already implemented in the physics pkg]
    \renewcommand{\Re}{\mathcal{R}} % Re symbol
    \renewcommand{\Im}{\mathcal{I}} % Im symbol
    \newcommand{\del}{\br{\nabla}}  % bold nabla symbol
    % frequently-used notations
    \newcommand{\rr}{\bm{r}}
    \newcommand{\pp}{\bm{p}}
    \newcommand{\LL}{\bm{L}}
    \renewcommand{\AA}{\bm{A}}
    \renewcommand{\dag}{\dagger}    % dagger symbol
    \newcommand{\diag}{\rm{diag}\ } % diagonal matrix notation
    \newcommand{\eps}{\varepsilon}
    \newcommand{\up}{\uparrow}
    \newcommand{\down}{\downarrow}
    % colors
    \newcommand{\blue}[1]{\textcolor{blue}{#1}}
    \newcommand{\red}[1]{\textcolor{red}{#1}}
    \newcommand{\green}[1]{\textcolor{green}{#1}}
    \newcommand{\purple}[1]{\textcolor{purple}{#1}}
    % others
    \catcode`\。=\active % replace 。with .
    \newcommand{。}{\ifmmode\text{.}\else .\fi} % replace 。with .
    \renewcommand{\emph}[1]{\CJKunderline*[textformat=\bfseries]{#1}}
    \newcommand{\schrodinger}{Schr\"{o}dinger}

\begin{document}

    \title{\LARGE \textbf{考研量子力学速查手册}\\ \textsc{Quick Ref for Quantum Mechanics}}
    \author{\Large \textit{Mingyu Zhu}}
    \date{\today}

    \setcounter{page}{1}
    \maketitle
    \thispagestyle{empty}
    \tableofcontents
    \pagebreak

    \section{一维问题}

% ===============
\subsection{无限深方势阱}

我们设无限深方势阱位于$0<x<a$,势阱中$V=0$,由驻波条件给出势阱中的波函数
\begin{equation}
    \psi_n(x) = \sqrt{\frac2a}\sin{k_n x}, \quad k_n = \frac{n\pi}{a},
\end{equation}
对应的能量为
\begin{equation}
    E_n = \frac{\hbar^2 k_n^2}{2m} = \frac{\hbar^2 \pi^2 n^2}{2ma^2}.
\end{equation}
注意:记忆能级不靠背诵,利用$p=\hbar k$及$E=p^2/2m$即可定出能级。

在求解如$x,p,x^2,p^2,H$等物理量的平均值时,注意:
\begin{itemize}
    \item 不少情况下,利用波函数的对称性及不同能级波函数的正交性可简化问题。
    \item 对于不变的势阱,$p,H$是守恒量,因此其平均值不随时间变化。
    \item 熟记二倍角公式:
    \begin{equation}
    \begin{aligned}
        \sin 2\alpha &= 2\sin\alpha\cos\alpha, \\
        \cos 2\alpha &= \cos^2\alpha - \sin^2\alpha = 2\cos^2\alpha - 1 = 1 - 2\sin^2\alpha.
    \end{aligned}
    \end{equation}
    \item 熟记积化和差公式:
    \begin{equation}
    \begin{aligned}
        2\cos\alpha\cos\beta &= \cos(\alpha-\beta) + \cos(\alpha+\beta), \\
        2\sin\alpha\sin\beta &= \cos(\alpha-\beta) - \cos(\alpha+\beta), \\
        2\sin\alpha\cos\beta &= \sin(\alpha+\beta) + \sin(\alpha-\beta).
    \end{aligned}
    \end{equation}
\end{itemize}

% ======================
\subsection{谐振子:坐标表象}

一维谐振子势为
\begin{equation}
    V(x) = \frac12 m \omega^2 x^2.
\end{equation}

置
\begin{equation}
    \alpha = \sqrt{\frac{m\omega}{\hbar}},
\end{equation}
解得谐振子的波函数为
\begin{equation}
    \psi(x) = \sqrt{\frac{\alpha}{\sqrt\pi 2^n n!}} H_n(\alpha x) e^{-\alpha^2 x^2/2},
\end{equation}
其中$H_n$为$n$阶Hermite函数。

能级为
\begin{equation}
    E_n = (n + \frac12) \hbar \omega.
\end{equation}


% ======================
\subsection{谐振子:Fock表象}

引入无量纲算符
\begin{equation}
    Q \coloneq \sqrt{\frac{m\omega}{\hbar}} x ,\quad P \coloneq \sqrt{\frac{1}{m\hbar\omega}} p,
\end{equation}
其对易子$[Q,P]=\ii$,
进一步定义算符
\begin{equation}
    a \coloneq \frac{1}{\sqrt{2}} (Q+\ii P) ,\quad a^\dag \coloneq \frac{1}{\sqrt{2}} (Q-\ii P),
\end{equation}
其对易子$[a,a^\dag]=1$,可得
\begin{equation}
    H = (a^\dag a + \frac12) \hbar\omega.
\end{equation}
由
\begin{equation}
    a^\dag a \ket{n} = n \ket{n}
\end{equation}
可见$a^\dag a$为谐振子的能级算符。

$a$和$a^\dag$分别对应谐振子的\emph{降算符}与\emph{升算符}。
由对易子$[a,a^\dag]=1$及Hamiltonian $H$的表达式可知:
\begin{equation}
    H a \ket{n} = (E_n - \hbar\omega) a \ket{n},\quad H a^\dag \ket{n} = (E_n + \hbar\omega) a^\dag \ket{n},
\end{equation}

$a$并非保留内积的幺正算符。
通过归一条件$\braket{n}{n}=\braket{n-1}{n-1}=1$可得
\begin{equation}
    a\ket{n} = \sqrt{n}\ket{n-1}.
\end{equation}
同理有
\begin{equation}
    a^\dag \ket{n} = \sqrt{n+1}\ket{n+1}.
\end{equation}

    \clearpage
    \section{算符对易关系}

% ===============
\subsection{基本思路}

\emph{基本对易关系}
\begin{equation}
    [x,p] = i \hbar.
\end{equation}

\emph{常用规则}
\begin{itemize}
    \item 交换取负:$[A,B] = -[B,A]$;
    \item ``加法分配律'':$[\alpha A+\beta B,C] = [\alpha A,C]+[\beta B,C]$;
    \item ``乘法分配律''(前推前,后推后):
    \begin{equation}
        [A, \blue{B}\red{C}] = [A,\blue{B}] \red{C} + \blue{B}[A,\red{C}].
    \end{equation}
\end{itemize}

\emph{含矢量对易子的处理方法}
\begin{itemize}
    \item 若对易子本身为矢量,可以考虑将其分解为各个方向的分量。\\
    例如:要证$\LL\times\pp+\pp\times\LL = 2\ii\hbar\pp$,分别证明其三个分量成立即可,以$x$方向为例:$(\LL\times\pp+\pp\times\LL)_x = [p_y,L_z]+[L_y,p_z] = 2\ii\hbar p_x$。
    \item 对易子中含$\rr^2,\pp^2,\LL^2$的,考虑拆为分量形式。\\
    例如:$\rr^2=x^2+y^2+z^2$。
    \item 对易子中矢量的内积与外积同样满足``乘法分配律'':
    \begin{equation}
    \begin{aligned}
        \relax
        [A, \blue{\bm{B}}\cdot\red{\bm{C}}] &= [A,\blue{\bm{B}}]\cdot\red{\bm{C}} + \blue{\bm{B}}\cdot[A,\red{\bm{C}}], \\
        [A, \blue{\bm{B}}\times\red{\bm{C}}] &= [A,\blue{\bm{B}}]\times\red{\bm{C}} + \blue{\bm{B}}\times[A,\red{\bm{C}}].
    \end{aligned}
    \end{equation}
    \item 外积的定义:
    \begin{equation}
        \bm{\blue{A}}\times\bm{\red{B}} =
        \begin{vmatrix}
            \hat{\bm{x}} & \hat{\bm{y}} & \hat{\bm{z}} \\
            \blue{A_x} & \blue{A_y} & \blue{A_z} \\
            \red{B_x} & \red{B_y} & \red{B_z}
        \end{vmatrix}
    \end{equation}
\end{itemize}

% ===============
\subsection{坐标、动量与角动量的对易子}

牢记以下对易关系:
\begin{equation}
\begin{aligned}
    \relax
    [r_i, p_j] &= \ii\hbar\delta_{ij}, \\
	[L_i, r_j] &= \ii\hbar\epsilon_{ijk} r_k, \\
    [L_i, p_j] &= \ii\hbar\epsilon_{ijk} p_k, \\
    [L_i, L_j] &= \ii\hbar\epsilon_{ijk} L_k,
\end{aligned}
\end{equation}
这里$\epsilon_{ijk}$是Levi-Civita符号,仅在$i\ne j\ne k$时不为零,在$ijk$为偶排列时为$1$,奇排列时为$-1$。

角动量的分量式有时能派用场:
\begin{equation}
    \LL = \rr\times\pp =
    \begin{vmatrix}
        \hat{\bm{x}} & \hat{\bm{y}} & \hat{\bm{z}} \\
        x & y & z \\
        p_x & p_y & p_z \\
    \end{vmatrix}
    = \underbrace{(y p_z - z p_y)}_{L_x} \hat{\bm{x}} + \underbrace{(z p_x - x p_z)}_{L_y} \hat{\bm{y}} + \underbrace{(x p_y - y p_x)}_{L_z} \hat{\bm{z}}.
\end{equation}

% ===============
\subsection{``幂次''公式}

我们设$[A, B] = C$,且$C$与$A$与$B$均对易。
容易证明``幂次''公式
\begin{equation}
\begin{aligned}
    \relax
    [A,B^n]
    &= [A,\blue{B}\cdot \red{B^{n-1}}] \\
    &= \blue{B} [A, \red{B^{n-1}}] + \underbrace{[A, \blue{B}]}_C \red{B^{n-1}} \\
    &= B^2 [A, B^{n-2}] + 2C B^{n-1} \\
    &= \cdots \\
    &= n C B^{n-1}.
\end{aligned}
\end{equation}

这一公式可以进一步推广至$[A,f(B)]$的情形。
我们假定$f(B)$是解析函数,于是
\begin{equation}
\begin{aligned}
    \relax
    [A,f(B)]
    &= \left[A, \sum_n \frac{f_n}{n!} B^n\right] \\
    &= \sum_n \frac{f_n}{n!} [A, B^n] \\
    &= C \sum_n \frac{f_n}{(n-1)!} B^{n-1} \\
    &= C \frac{\pd f(B)}{\pd B}.
\end{aligned}
\end{equation}

对于$A,B$是坐标与动量(的函数)这一特殊情形,有
\begin{equation}
\begin{aligned}
    \relax
    [\rr, f(\rr,\pp)] = \ii\hbar\nabla_{\pp} f(\rr,\pp), \\
    [\pp, f(\rr,\pp)] = -\ii\hbar\nabla_{\rr} f(\rr,\pp).
\end{aligned}
\end{equation}
取一例:
\begin{equation}
    [\pp, \frac{1}{r}] = [p_x \hat{\bm{x}} + p_y \hat{\bm{y}} + p_z \hat{\bm{z}}, \frac{1}{r}] = -\ii\hbar \nabla \frac{1}{r} = \ii\hbar \frac{\hat{\bm{r}}}{r^2}.
\end{equation}
这里的$\nabla_{\rr}, \nabla_{\pp}$应当理解为\emph{``对算符求导数''}。
虽然看起来与$\rr,\pp$算符在彼此的表象中的形式
\begin{equation}
    \rr^{(\pp)}\equiv\ii\hbar\nabla_{\pp},
    \quad \pp^{(\rr)}\equiv-\ii\hbar\nabla_{\rr}
\end{equation}
完全一样,但本质完全不同!

    \clearpage
    \section{时间演化}

\emph{Heisenberg方程}
\begin{equation}
    \frac{\dd \expval{A}}{\dd t} = \frac{1}{i\hbar}\expval{[A,H]} + \expval{\dot{A}}.
\end{equation}

利用Heisenberg方程可导出以下推论:

\emph{Ehrenfest定理}:
对于
\begin{equation}
    H = \frac{\pp^2}{2m} + V(\rr),
\end{equation}
有
\begin{equation}
    \frac{\dd \expval{\rr}}{\dd t} = \expval{\pp}/m,
\end{equation}
\begin{equation}
    \frac{\dd \expval{\pp}}{\dd t} = -\expval{\nabla V(\rr)}.
\end{equation}
这与经典力学的Newton方程是相对应的。

\emph{Feynman-Hellmann定理}:
设体系的Hamiltonian$H(\lambda)$由参数$\lambda$控制,而$E_n(\lambda)$为其能级,有
\begin{equation}
    \frac{\pd E_n}{\pd \lambda} = \expval{\frac{\pd H}{\pd \lambda}}_n.
\end{equation}

\emph{Virial定理}:
考察$\rr\cdot\pp$的期望值随时间的变化,由于定态的力学量平均值不随时间变化,因此其必为零,可得
\begin{equation}
    2\expval{T} = \expval{\rr\cdot\nabla V(\rr)}.
\end{equation}

    \clearpage
    \section{全同粒子}

\emph{Boson玻色子}:$s$为整数,波函数满足交换对称性。

\emph{Fermion费米子}:$s$为半奇数,波函数满足交换反对称性,因而有Pauli不相容原理。

一个常见的问题是计算Bose, Fermi与经典体系粒子的组合数:
$n$个全同粒子处于$m$个可能的单粒子态,对于Bose子, Fermi子与经典粒子,可能有几种组合方式?

对于经典粒子,粒子是完全可分辨的,因此粒子间互不相关,可能的组合数有
\begin{equation}
    N_{\rm{classical}} = m^n.
\end{equation}

对于Fermi子,粒子不可分辨,可以用隔板法求取可能的组合数:
设置$m-1$个隔板,与$n$个粒子混合排列,这样$n$个粒子就被隔板分至$m$个态上,有$(n+m-1)!$种方式。
隔板与粒子分别不可分辨,因此除以各自排列数$(m-1)!$与$n!$
\begin{equation}
    N_{\rm{Fermi}} = \frac{(n+m-1)!}{(m-1)! n!}.
\end{equation}

对于Bose子,粒子不可分辨,且是互斥的,于是从$m\ (m>n)$个态中选取$n$个分别放入一个粒子即可,可能的组合数:
\begin{equation}
    N_{\rm{Bose}} = P_m^n = \frac{m!}{n!(m-n)!}.
\end{equation}
    \clearpage
    \section{中心力场中的氢原子}

中心力场中的氢原子,选取质心坐标和自然单位制,波函数为
\begin{equation}
    \psi_{nlm}(\rr) = Y_{lm}(\theta,\phi) R_{nl}(r),
\end{equation}

\begin{itemize}
    \item 角动量平方的期望值$\expval{\LL^2}=l(l+1)\hbar^2$;
    \item 角动量$z$分量的期望值$\expval{L_z}=m\hbar$。
\end{itemize}

计算$r^n$物理量的期望值时,可以采用如下的策略:
\begin{equation}
\begin{aligned}
    \expval{r^n}
    &= \frac{\int r^2 \dd r \dd\Omega \ Y_{lm}^*(\theta,\phi) r^n Y_{lm}(\theta,\phi) R_{nl}(r)^2}{\int r^2 \dd r \dd\Omega \ Y_{lm}^*(\theta,\phi) Y_{lm}(\theta,\phi) R_{nl}(r)^2} \\
    &= \frac{\int\dd r \ r^{n+2} R_{nl}(r)^2}{\int\dd r \ r^2 R_{nl}(r)^2},
\end{aligned}
\end{equation}
其中$R_{nl}(r)$是多项式$P(r)$与$\ee^{-r/na_0}$的乘积,因此问题最后可以归结为求解一个重要积分:
\begin{equation}
    \int_0^\infty \dd t \ t^{N} \ee^{-t} = \Gamma(N+1) = N!,
\end{equation}
读者应牢记。

    \clearpage
    \section{Landau能级}

粒子在电磁场中的Hamiltonian:
\begin{equation}
    H = \frac{1}{2m}\left(\pp-\frac{q}{c}\AA\right)^2 + q\phi,
\end{equation}
其中$\pp$是正则动量,$\AA$是磁场矢势,$\phi$是电势。
按电动力学,电场与磁场用势能表示为:
\begin{equation}
    \bm{E} = -\frac1c \pd_t \AA - \nabla \phi, \quad
    \bm{B} = \nabla \times \AA.
\end{equation}

考虑处于沿$z$轴的静磁场中的粒子,选取Landau规范
\begin{equation}
    A_x = -By, \quad A_y = A_z = 0,
\end{equation}
可得Hamiltonian:
\begin{equation}
    H = \frac{1}{2m}\left[ (p_x-qBy/c)^2 + p_y^2 + p_z^2 \right].
\end{equation}
鉴于$[p_x,H]=[p_z,H]=0$,$p_x,p_z$与$H$有共同本征态,取为
\begin{equation}
    \psi(x,y,z) = \ee^{\ii p_x x/\hbar} \ee^{\ii p_z z/\hbar} \phi_{p_x,p_z}(y),
\end{equation}
有
\begin{equation}
    H \phi(y) = \frac{1}{2m}\left[ (p_x-qB\hat{y}/c)^2 + \hat{p}_y^2 + p_z^2 \right] \phi(y),
\end{equation}
注意这里对算符$y,p_y$作了帽子记号,以区别于$p_x,p_z$(其作为$\phi(y)$的参数出现,是定值)。
置
\begin{equation}
    y_0 = \frac{p_x c}{qB}, \quad \omega = \frac{qB}{mc},
\end{equation}
于是Hamiltonian成为
\begin{equation}
    \frac12 m\omega^2 (y-y_0)^2 + \frac{1}{2m}p_y^2 + \frac{1}{2m}p_z^2.
\end{equation}
可见除去$z$方向动能,这是一个原点位于$y=y_0$的一维谐振子,能级表达式为
\begin{equation}
    E_n = \hbar \omega \left( n + \frac12 \right).
\end{equation}

    \clearpage
    \section{自旋-1/2体系}

% ===============
\subsection{Pauli表象}

对于自旋-1/2体系,Pauli表象选取$s_z=\pm\frac12\hbar$(即$\ket{\up},\ket{\down}$)的本征态作为描述系统状态的基矢。

定义满足
\begin{equation}
    \bm{s} = \frac{\hbar}{2}\bm{\sigma}
\end{equation}
的单位化Pauli自旋算符$\bm{\sigma}$,其各分量表为
\begin{equation}
    \sigma_z = \begin{pmatrix}1& 0\\0&-1\end{pmatrix}, \quad
    \sigma_x = \begin{pmatrix}0& 1\\1& 0\end{pmatrix}, \quad
    \sigma_y = \begin{pmatrix}0&-\ii\\ \ii& 0\end{pmatrix}.
\end{equation}
其满足
\begin{equation}
    \sigma_i^2 = 1, \quad
    \sigma_i\sigma_j = \ii \epsilon_{ijk}\sigma_k \quad (i\ne j)
\end{equation}

在计算单自旋-1/2粒子量子态的含时演化时,一般情形需直接求解二分量\schrodinger 方程:
\begin{equation}
    \ii\hbar\begin{pmatrix}\dot{\psi}_+\\ \dot{\psi}_-\end{pmatrix} =
    H \begin{pmatrix}\psi_+\\ \psi_-\end{pmatrix}.
\end{equation}
但若Hamiltonian能表为$\bm{\sigma}$的分量的线性组合(例如:$H=E_0\sigma_i$),可利用结论
\begin{equation}
    \ee^{\ii \bm{\alpha}\cdot\bm{\sigma}} = \cos(\alpha) + \ii \sin(\alpha) (\hat{\bm{\alpha}}\cdot\bm{\sigma}),
\end{equation}
而无需变换至$\sigma_i$表象。
该结论可利用$\ee^{\ii \bm{\alpha}\cdot\bm{\sigma}}$的Taylor展开式证明,类似的证明亦是考点之一。

% ===============
\subsection{双自旋-1/2粒子}

描述双自旋-1/2粒子的量子态,有两种表象:
\begin{itemize}
    \item \emph{非耦合表象}:选取$s_{1z}, s_{2z}$为好量子数;
    \item \emph{耦合表象}:设$\bm{S}=\bm{s_1}+\bm{s_2}$为合角动量,选取$S^2, S_z$为好量子数。
\end{itemize}

非耦合表象选取的基底为
\begin{equation}
    \ket{\sigma_{1z}\sigma_{2z}} =\quad \ket{\up\up}, \quad \ket{\up\down}, \quad \ket{\down\up}, \quad \ket{\down\down}.
\end{equation}
直积态的矩阵表示可用Kroncker积计算,注意Kronecker积的运算规则:
\begin{equation}
    A \otimes B =
    \begin{pmatrix}
        A_{11}B & A_{12}B & \cdots \\
        A_{21}B & A_{22}B & \cdots \\
        \vdots  & \vdots  & \ddots \\
    \end{pmatrix}
\end{equation}

耦合表象选取的基底为
\begin{equation}
    \ket{\sigma \sigma_z} = \quad \underbrace{\ket{11}, \quad \ket{10}, \quad \ket{1-1}}_{\text{triplet}}, \quad \underbrace{\ket{00}}_{\text{singlet}}.
\end{equation}
由于
\begin{equation}
    \bm{S}^2 = \frac{\hbar^2}{2}(3+\bm{\sigma_1}\cdot\bm{\sigma_2}),
\end{equation}
可在非耦合表象下求取$\bm{\sigma_1}\cdot\bm{\sigma_2}$的本征值为$1,1,-3,1$,分别对应
\begin{equation}
    \ket{11}=\ket{\up\up},\quad \ket{10}=\frac{1}{\sqrt{2}}(\ket{\up\down}+\ket{\down\up}),\quad \ket{1-1}=\ket{\down\down}, \quad \ket{00}=\frac{1}{\sqrt{2}}(\ket{\up\down}-\ket{\down\up}).
\end{equation}

对于两个全同Bose和Fermi子,由于对称性限制,无法取以上所有的态!
全同Bose子仅能取三重态$\ket{11}, \ket{10}, \ket{1-1}$,而全同Fermi子只能取单重态$\ket{00}$.

如果一个双粒子体系组成的量子态能表为单粒子态的直积,则称为可分离态,否则,称为纠缠态。
Bell基是一组完备的双粒子自旋-1/2系统的基底,是四个最大纠缠态:
\begin{equation}
    \ket{\psi^{\pm}}=\frac{1}{\sqrt{2}}(\ket{\up\down}\pm\ket{\down\up}),\quad
    \ket{\phi^{\pm}}=\frac{1}{\sqrt{2}}(\ket{\up\up}\pm\ket{\down\down}).
\end{equation}

    \clearpage
    \section{角动量}

% ===============
\subsection{升降算符}

升降算符定义为
\begin{equation}
    L_\pm = L_x \pm \ii L_y,
\end{equation}
从角动量算符对易关系可得升降算符满足的关系:
\begin{equation}
\begin{aligned}
    L_\pm L_\mp = L^2 - L_z^2 \pm \hbar L_z, \\
    L_x^2 + L_y^2 = L^2 - L_z^2 = \frac12 \left(L_\pm L_\mp + L_\mp L_\pm \right), \\
    [L_z, L_\pm] = \pm \hbar L_\pm, \\
    [L^2, L_\pm] = [L^2, L_z] = 0.
\end{aligned}
\end{equation}

升降算符对角动量态的作用效果:
\begin{equation}
\begin{aligned}
    L_+ \ket{lm} &= \sqrt{l(l + 1) - m(m + 1)} \ket{l, m+1},\\
    L_- \ket{lm} &= \sqrt{l(l + 1) - m(m - 1)} \ket{l, m-1},
\end{aligned}
\end{equation}
这里矩阵元的正负号可利用$L_+\ket{ll}=0$与$L_-\ket{l-l}=0$助记。

有时需要求解$L_x, L_y$的矩阵元,可利用
\begin{equation}
    L_x = \frac12 (L_+ + L_-), \quad L_y = \frac{1}{2\ii} (L_+ - L_-).
\end{equation}

% ===============
\subsection{角动量的耦合}

角动量的\emph{非耦合表象}:选取$L_1^2, L_2^2, L_{1z}, L_{2z}$力学量,以对应的$l_1,l_2,m_1,m_2$作为好量子数。
$m_1, m_2$的取值范围:
\begin{equation}
    -l_1 \le m_1 \le l_1,\quad -l_2 \le m_2 \le l_2.
\end{equation}

\emph{耦合表象}:选取$L_1^2, L_2^2, L^2=(\bm{L}_1 +\bm{L}_2)^2, L_z = L_{1z} + L_{2z}$力学量,以对应的$l_1,l_2,l,m$作为好量子数。
$l,m$的取值范围:
\begin{equation}
    \abs{l_1-l_2}\le l \le l_1+l_2,\quad -l \le m \le l.
\end{equation}

两个表象之间的变换矩阵元:
\begin{equation}
    \braket{l_1m_1l_2m_2}{l_1l_2lm} = C_{l_1m_1l_2m_2}^{lm},
\end{equation}
称为Clebsch-Gordan (CG)系数。

    \clearpage
    \section{定态微扰}

我们设体系的Hamiltonian为$H=H_0+H'$,未扰Hamiltonian $H_0$的第$m$个本征态表为$\ket{\psi_m^{(0)}}$,对应能级$E_m^{(0)}$。
$\ket{\psi_m}$能量与本征态的$i$阶微扰修正用$E_m^{(i)}$和
\begin{equation}
    \ket{\psi_m^{(i)}} = \sum_n c_{mn}^{(i)} \ket{\psi_n^{(0)}}
\end{equation}
表示。这样,修正后能量与本征态表为
\begin{equation}
    E_m=E_m^{(0)}+E_m^{(1)}+\cdots, \quad \ket{\psi_m}=\ket{\psi_m^{(0)}}+\ket{\psi_m^{(1)}}+\cdots.
\end{equation}

\begin{itemize}
    \item \emph{一阶定态微扰的能量修正值}
    \begin{equation}
        E_m^{(1)} = \bra{\psi_m^{(0)}}H'\ket{\psi_m^{(0)}} = H'_{mm}.
    \end{equation}
    \item \emph{一阶定态微扰的本征态修正}
    \begin{equation}
        \ket{\psi_m^{(1)}} = \sum_n c_{mn}^{(1)} \ket{\psi_m^{(0)}},
    \end{equation}
    \begin{equation}
        c_{mn}^{(1)} = \begin{cases}
            \frac{H'_{nm}}{E_m^{(0)}-E_n^{(0)}} & \text{for } m \neq n, \\
            0 & \text{for } m = n.
        \end{cases}
    \end{equation}
    \item \emph{二阶定态微扰的能量修正值}
    \begin{equation}
        E_m^{(2)} = \sum_{n\ne m} \frac{\abs{H'_{mn}}^2}{E_m^{(0)}-E_n^{(0)}}.
    \end{equation}
\end{itemize}

对于未扰Hamiltonian存在简并能量本征态的情况下,不能直接应用以上公式,需要先行选择一组特殊的未扰基矢$\ket{\psi_m^{(0)}}$。
设原先的一组未扰简并能量本征态为$\ket{\phi_i}$($i=1,\cdots,N$),求出关于本征值$E^{(1)}$的久期(secular)方程
\begin{equation}
    \det \left(H' - E^{(1)}\mathbf{I} \right) =
    \det \begin{pmatrix}
        H'_{11}-E^{(1)} & H'_{12} & \cdots & H'_{1N} \\
        H'_{21} & H'_{22}-E^{(1)} & \cdots & H'_{2N} \\
        \vdots & \vdots & \ddots & \vdots \\
    \end{pmatrix} = 0
\end{equation}
对应的解$E_m^{(1)}$,并求解出对应于这些本征值的本征矢$\psi_m^{(0)}$,即可作为未扰基矢,对应的一阶能量修正已由$E_m^{(1)}$给出。


\end{document}